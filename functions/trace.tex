\qpreamble{Trace the call of the \texttt{myfun()} function and write what will be printed. Be sure to accurately show linefeeds.}

\qversion{a}{

\code{
def myfun(c,n): \\
\mytab    print(\sq--\sq,end=\sq{} \sq) \\
\mytab    i = 0 \\
\mytab    while i < n: \\
\mytab\mytab        print(f\sq{c}-{c}\sq,end=\sq{} \sq) \\
\mytab\mytab        i = i + 1 \\
 \mytab   print(\sq--\sq)
}

\texttt{myfun(\sq X\sq,4)}
}

\qversion{b}{

\code{
def myfun(a,n): \\
\mytab    i = 0 \\
\mytab    while i < n: \\
\mytab\mytab        print(f\sq-{a}-\sq,end=\sq{} \sq) \\
\mytab\mytab        i = i + 1 \\
\mytab    print()
}

\texttt{myfun(\sq\%\sq,4)}
}

\qversion{c}{

\code{
def myfun(c,n): \\
\mytab    i = 0 \\
\mytab    while i < n: \\
\mytab\mytab        print(f\sq{c}{c}\sq,end=\sq{} \sq) \\
\mytab\mytab       i = i + 1 \\
\mytab    print() 
}

\texttt{myfun(\textquotesingle A\textquotesingle,4)}
}

\qversion{d}{
\code{
def myfun(c,n): \\
\mytab    print(f\sq{c}{c}\sq,end=\sq{} \sq) \\
\mytab    i = 0 \\
\mytab    while i < n: \\
\mytab\mytab        print(f\sq--\sq,end=\sq{} \sq) \\
\mytab\mytab       i = i + 1 \\
\mytab    print(f\sq{c}{c}\sq)
}

\texttt{myfun(\textquotesingle\#\textquotesingle,4)}
}

\qpostamble{
    \lines{6}   
}