
\qpreamble{
Define a function as described below. After defining the function, \textbf{call it with your choice of arguments. Make the call part of an assignment statement and print the assigned variable}. \\
}

\qversion{a}{Define a function called \textit{add\_all}. It has 3 parameters. In the function, add the values passed to the function and return the results.
}

\qversion{b}{
Define a function called \textit{product}. It has 3 parameters. In the function, multiply the 3 passed values. Return the results.
}

\qversion{c}{
Define a function called \textit{square}. It has 1 parameter. In the function, square the passed value (i.e. multiply it by itself). Return the results.
}

\qversion{d}{
Define a function called \textit{remove\_dec}. It has 1 parameter of type float. In the function, remove the decimal portion of the passed float by dividing it by 1 (e.g. //1 ). Return the results.
}

\qversion{e}{
Define a function called \textit{percent}. It has 1 parameter. In the function, multiply the passed value by 100. Return the results.
}

\qversion{f}{
Define a function called \textit{add10}. It has 1 parameter. In the function, add 10 to the passed value. Return the results. \\
}

\qpostamble{
    \vspace{5mm}
    \lines{8}
}