
\qpreamble{Consider the following lines of code from the GupPy syntax:}

\qversion{a}{
\code{
Tile(YELLOW).place(x,2) \\
Tile(RANDOM).place(x,2) \\
x = x + 2 \\
x = 5
}
The above code contains all lines necessary to place 2 tiles with one to the right of the other, but IN THE WRONG ORDER.

}

\qversion{b}{
\code {
y = y + 2 \\
Tile(RANDOM).place(6,y) \\
Tile(PINK).place(6,y) \\
y = 5 
}
The above code contains all lines necessary to place 2 tiles with one above the other, but IN THE WRONG ORDER. 

}

\qversion{c}{
\code {
Tile(PINK).place(x,5) \\
x = 6 \\
Tile(YELLOW).place(x,5) \\
x = x - 2
}
The above code contains all lines necessary to place 2 tiles with one to the left of the other, but IN THE WRONG ORDER. 

}

\qversion{d}{
\code {
y = y - 2 \\
y = 6 \\
Tile(PINK).place(3,y) \\
Tile(YELLOW).place(3,y) 
}
The above code contains all lines necessary to place 2 tiles with one below the other, but IN THE WRONG ORDER. 

}

\qpostamble{
\vspace{3mm}
\textbf{Rewrite the code in the correct order} to place the tiles as indicated.
    \lines{4}   % 4 lines at default mm apart
}