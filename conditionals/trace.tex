\qpreamble{Consider the code below.}

\qversion{a} {
\vspace{3mm}
\code{
   1  x = \rule{28mm}{0.25mm} \\
   2  y = 6 \\
   3  if x <= y: \\
   4  \mytab    z = x \\
   5  \mytab    print(x)  \\
   6  else: \\
   7  \mytab    z = y \\
   8   \mytab   print(y)
}
%\vspace{4mm}
\noindent{Write a value for x at line \#1 that would result in the execution of code block at line \#4 and \#5.}
}

\qversion{b} {
\code{
   1  x = roll()\\
   2  if x < 4:\\
   3  \mytab   Tile(RANDOM).place(x,6)\\
   4  \mytab    x = x  2\\
   5  else:\\
   6  \mytab    Tile(RANDOM).place(x,10)\\
   7   \mytab   x = x - 2
}
\vspace{5mm}
\noindent{State a roll() value for line \#1 that would result in the execution of code block at line \#3 and \#4.}
\vspace{5mm}

roll() = \rule{28mm}{0.25mm}

}

\qversion{c} {
\code{
     1  color = \rule{28mm}{0.25mm} \\
     2  y = 2\\
     3  if color == PINK:\\
     4  \mytab    Tile(color).place(0,y)\\
     5  \mytab    Tile(YELLOW).place(0,y+2)\\
     6  else:\\
     7  \mytab    Tile(color).place(0,y)\\
     8  \mytab    Tile(PINK).place(0,y+2
}
\vspace{5mm}
\noindent{State a color value for line \#1 that would result in the execution of the code block at line \#4 and \#5.
}
\vspace{5mm}

color = \rule{28mm}{0.25mm}
}

\qversion{d} {
\code{
     1  x = roll() + roll()\\
     2  y = roll() + roll()\\
     3  if x > 10: \\
     4  \mytab    x = 10 \\
     5  else:\\
     6  \mytab    x = 6\\
     7  if y == 12:\\
     8  \mytab    y = 10 
}
\pgfmathsetmacro{\A}{random(8,12)}
\pgfmathsetmacro{\B}{random(10,12)}
\vspace{5mm}
\noindent{In the code above, circle the lines of code that will be executed if the value x is initialized to {\A} and the value y is initialize to {\B}.
}
}

\qversion{e} {
\code{
     1  x = roll() * 2\\
     2  y = roll() * 2\\
     3  if x < 6: \\
     4  \mytab     dx = 2\\
     5  else:\\
     6  \mytab     dx = -2\\
     7  if y < 6:\\
     8  \mytab     dy = 2\\
     9  else:\\
     10  \mytab    dy = -2
}
\pgfmathsetmacro{\C}{random(1,6)*2}
\pgfmathsetmacro{\D}{random(1,6)*2}
\vspace{5mm}
\noindent{Circle the lines of code above that will be executed if the value x is initialized to {\C} and the value y is initialize to {\D}.}
}

\qversion{f} {
\code{
   1  x = \rule{28mm}{0.25mm} \\
   2  y = 6 \\
   3  if x <= y: \\
   4  \mytab    z = x \\
   5  \mytab    print(x)  \\
   6  else: \\
   7  \mytab    z = y \\
   8  \mytab    print(y) 
}
\vspace{5mm}
\noindent{Write a value for x in the blank at line \#1 that would result in the execution of the code block at line \#7 and \#8.}
}

\qversion{g} {
\code{
   1  x = roll()\\
   2  if x < 4:\\
   3  \mytab    Tile(RANDOM).place(x,6)\\
   4  \mytab    x = x + 2\\
   5  else:\\
   6  \mytab    Tile(RANDOM).place(x,10)\\
   7  \mytab    x = x - 2
}
\vspace{5mm}
\noindent{State a roll() value for line \#1 that would result in the execution of the code block at line \#6 and \#7.}
\vspace{5mm}

roll() = \rule{28mm}{0.25mm}
}

\qversion{h} {
\code{
     1  color = \rule{28mm}{0.25mm} \\
     2  y = 2\\
     3  if color == PINK:\\
     4  \mytab    Tile(color).place(0,y)\\
     5  \mytab    Tile(YELLOW).place(0,y+2)\\
     6  else:\\
     7  \mytab    Tile(color).place(0,y)\\
     8  \mytab    Tile(PINK).place(0,y+2
}
\vspace{5mm}
\noindent{State a color value for line \#1 that would result in the execution of the code block at line \#7 and \#8.
}
}

\qpostamble{}